%%% Local Variables: 
%%% mode: latex
%%% TeX-master: t
%%% End: 

%% to generate notes
%\documentclass[]{article}
%\usepackage{beamerarticle}

\documentclass[ignorenonframetext]{beamer}
\usetheme{default}
\usefonttheme{serif}

\usepackage[brazil]{babel}
\usepackage[utf8]{inputenc}
\usepackage{graphicx}
\usepackage{hyperref}
\usepackage{tikz}

\def\vertex{vértice}
\def\vertices{vértices}
\def\oslist{GNU$/$Linux}
\def\graphdef{Graph Definition}

\begin{document}

\begin{frame}
\title{CNAOS}
\author{Adriano J. Holanda}
\maketitle
\end{frame}

\section{Introdução}
\label{sec:intro}

\begin{frame}[plain]
\title{\LARGE Introdução}
\author{}
\date{\empty}
\maketitle
\end{frame}


\begin{frame}
\includegraphics[width=\textwidth]{img/thesaurus_paper_header}
\end{frame}

\begin{frame}
\includegraphics[width=\textwidth]{img/thesaurus_paper_fig1}
\end{frame}

\begin{frame}
\includegraphics[width=\textwidth]{img/thesaurus_paper_fig2}
\end{frame}

% \begin{frame}
% \includegraphics[width=\textwidth]{img/thesaurus_paper_fig3}
% \end{frame}

% \begin{frame}
% \includegraphics[width=\textwidth]{img/thesaurus_paper_fig4}
% \end{frame}


 \begin{frame}
 \includegraphics[width=\textwidth]{img/culinary_paper_header}
 \end{frame}

% \begin{frame}
% \includegraphics[width=\textwidth]{img/culinary_paper_tab1}
% \end{frame}

\begin{frame}
\includegraphics[width=\textwidth]{img/culinary_paper_fig1}
\end{frame}

% \begin{frame}
% \includegraphics[width=\textwidth]{img/culinary_paper_fig2}
% \end{frame}

\section{Objetivo}

\begin{frame}{Objetivo}

\begin{itemize}
\item Analisar as relações nas
redes formadas pelo conjunto de objetos funcionais (funções) que
compõem alguns sistemas operacionais, tendo como parâmetros principais
: 
\begin{enumerate}
\item A distribuição
dos graus; 
\item Centralidade dos \vertex{}s;
\item Índice de agrupamento.
\end{enumerate}
\end{itemize}
\end{frame}

\section{Metodologia}

\begin{frame}[plain]

\title{\LARGE Metodologia}
\author{}
\date{\empty}
\maketitle
\end{frame}

\subsection{Conjunto de dados}

\begin{frame}{Conjunto de dados}
\begin{block}{Lista:}
\oslist
\end{block}

\begin{block}{Justificativa:}
\begin{itemize}
\item Acesso ao código fonte;
\item Utilização em larga escala ({\footnotesize GNU/Linux, FreeBSD, NetBSD});
\item Arquitetura baseada em micronúcleo ({\footnotesize Minix3}).
\end{itemize}
\end{block}

\end{frame}

\begin{frame}[fragile]
\frametitle{Formação do grafo}
\small

\graphdef

\hrule
\bigskip
\begin{columns}
\begin{column}{0.25\textwidth}
programa.c
\begin{verbatim}
void a() {
     b();
     c();
}
int b() {
    d();
    return 0;
}
void c() {
     b();
}
void d();
\end{verbatim}
\end{column}
{\LARGE $\Rightarrow$}\hspace{0.25cm}
\begin{column}{0.45\textwidth}

  \begin{tikzpicture}[mynode/.style={circle,draw},myedge/.style={->,>=latex}]
    \node[mynode] (a) {\scriptsize\tt a()};
    \node[mynode] (b) {\scriptsize\tt b()};
    \node[mynode] (c) {\scriptsize\tt c()};
    \node[mynode] (d) {\scriptsize\tt d()};
    \draw[myedge] (a) -- (b);
    \draw[myedge] (a) -- (c);
    \draw[myedge] (c) -- (d);
  \end{tikzpicture}

\end{column}
\end{columns}

\end{frame}

\begin{frame}

\graphdef

\begin{block}{Exemplo -- parte código do escalonador ({\tt sched.c}) do GNU/Linux}
  \usetikzlibrary{shapes,positioning}
\begin{tikzpicture}[mynode/.style={ellipse,draw},myedge/.style={->,>=latex}]
  \node[mynode] (resched) {\scriptsize\tt resched\_cpu};
  \node[mynode] (rq) [below=of resched] {\scriptsize\tt cpu\_rq};
  \node[mynode] (task) [below left=of resched] {\scriptsize\tt resched\_task};
  \node[mynode] (spin) [below left=of task] {\scriptsize\tt assert\_spin\_locked};
  \node[mynode] (smpmb) [below=of task] {\scriptsize\tt smp\_mb};
  \node[mynode] (irq) [below right=of task] {\scriptsize\tt spin\_locked\_irqsave};
  \draw[myedge] (resched) -- (rq);
  \draw[myedge] (resched) -- (task);
  \draw[myedge] (task) -- (smpmb);
  \draw[myedge] (task) -- (spin);
  \draw[myedge] (task) -- (irq);
\end{tikzpicture}

\end{block}

\end{frame}

\section{Medidas}

\begin{frame}{Medidas a serem realizadas}

\begin{block}{Medidas}
\begin{itemize}
  \item Distribuição dos graus dos \vertices{};
  \item Centralidade;
  \item Agrupamento.
 \end{itemize}
\end{block}
\end{frame}

\subsubsection{Distribuição do grau}

\begin{frame}{Distribuição dos graus dos nós}
\small
\begin{columns}
\begin{column}{0.5\textwidth}
\begin{block}{Grau de saída}
Número de arcos divergentes do \vertex.\\
\kouteq
\kavgouteq
\end{block}
\end{column}

\begin{column}{0.5\textwidth}
\begin{block}{Grau de entrada}
Número de arcos convergentes ao \vertex.\\
\kineq
\kavgineq
\end{block}
\end{column}
\end{columns}

\begin{block}{Grau médio total}
\kavgeq
\end{block}


\end{frame}

\begin{frame}{Probabilidade}
\probeq
\includegraphics[width=\textwidth]{img/prob_dist-culinary}
\cite{kinouchi}
\end{frame}

\subsubsection{Centralidade}

\begin{frame}{Centralidade}

\begin{block}{Grau}
\degcentreq
\end{block}
\begin{block}{Intermediação ({\em betweeness})}
\betcentreq
\noindent onde $\sigma$ -- número de caminhos geodésicos.
\end{block}
\begin{block}{Proximidade ({\em closeness})}
\closecentreq
\noindent onde $d_G$ -- distância geodésica.
\end{block}

\end{frame}

\begin{frame}{Índice de agrupamento}
\begin{columns}
\begin{column}{0.65\textwidth}
\footnotesize
\begin{block}{Triângulo}
\only<1>{
\[\delta(v) = 
\begin{cases}
1 &if\ \exists \{u,v\} \in E : \{v,w\}\in E \land \{w,u\} \in E\\
0 & caso\ contr\acute{a}rio
\end{cases}
\]
}
$\delta(G) = 1\slash 3 \sum_{v\in V}\delta(v)$\\
\includegraphics[scale=0.5]{img/triangle}
\end{block}
\end{column}

\begin{column}{0.45\textwidth}
\only<2->{
\begin{block}{Tripla}
$\tau(v) = {deg(v)\choose 2}$
$\tau(G) = \sum_{v\in V}\tau(v)$

\includegraphics[scale=0.45]{img/triple}
\end{block}
}
\end{column}

\end{columns}
\only<3>{
\begin{block}{Índice de agrupamento}
$\langle c(v) \rangle = \delta(v)\slash \tau(v)$ \hspace{0.5cm}\\
$\langle c(G) \rangle = \sum_{v\in V^\prime}\delta(v)\slash \tau(v)$\\
\noindent {\footnotesize $V^\prime$ -- conjunto de \vertex{}s com ${deg(v) \geq 2}$}.
\end{block}
}
\end{frame}

\begin{frame}{Análise dos resultados}

\begin{block}{Implicações}
\begin{itemize}
\item Distribuição cumulativa $\Rightarrow$ {\em fitting} das curvas,
análise comparativa entre os conjuntos do melhor {\em fitting};
\item Centralidade $\Rightarrow$ análise comparativa das funções no
conjunto de dados de cada sistema;
\item Índice de agrupamento $\Rightarrow$ existência de ciclos e grau de espalhamento.
\end{itemize}
\end{block}

\end{frame}

\section{Subprojetos}
\label{sec:subprojs}

\begin{frame}{Subprojetos}
\begin{block}{Enfoque}
\begin{enumerate}
\item \time\label{subproj:time} $\rightarrow$ GNU/Linux v1.0, v1.1,
v1.2, v1.3, v2.0, v2.1, v2.2, v2.3, v2.4, v2.5 e v2.6;
\item \projeto\label{subproj:projeto} $\rightarrow$ FreeBSD, GNU/Linux
e NetBSD;
\item \arquitetura\label{subproj:architecture} $\rightarrow$ sistemas
item~\ref{subproj:projeto} (monolíticos) e Minix3 (baseado em micronúcleo).
\end{enumerate}
\end{block}

\begin{block}{Cronograma}
\begin{center}
\begin{tabular}[h]{cc}\hline\hline
{\bf Ano} & {\bf Subprojeto} \\\hline
2011 -- 2013  & \ref{subproj:time} \\
2013 -- 2015 & \ref{subproj:projeto}\\
2016, 2017   & \ref{subproj:architecture}\\\hline\hline
\end{tabular}
\end{center}
\end{block}
\end{frame}

\begin{frame}{Referências}
\begin{thebibliography}{4}

\bibitem{holanda}[Holanda, 2004]
Adriano J. Holanda {\em et al.}
\newblock Physica A v. 344, 2004.

\bibitem{kinouchi}[Knouchi, 2008]
Osame Kinouchi {\em et al.}
\newblock New J. Phys. n. 7, 2008.
\end{thebibliography}
\end{frame}

\end{document}
